
%A global event reconstruction is performed using the particle-flow (PF) \cite{CMS-PAS-PFT-09-001, ParticleFlow} algorithm.
%, which optimally combines the information from all subdetectors and 
%produces a list of stable particles (muons, electrons, photons, 
%charged and neutral hadrons). 


\section{Event reconstruction}

The reconstructed interaction vertex with the largest value of $\sum_{i} p_\mathrm{T}^{i2}$, where $p_\mathrm{T}^{i}$ is the transverse momentum of the $i^\mathrm{th}$ track 
associated with the vertex, is selected as the primary event vertex. The offline selection requires all events to have at least one primary vertex reconstructed within a 24\,cm window along the z-axis around the nominal interaction point, and a transverse distance from the nominal interaction region less than 2\,cm.

The particle-flow (PF) \cite{Sirunyan:2017ulk} algorithm aims to reconstruct all the physics objects described in this section. At large Lorentz boosts, the two b quarks from the Higgs boson decay may produce jets that overlap and make their individual reconstruction difficult. In this search, large-area jets clustered from PF candidates using the Cambridge--Aachen algorithm~\cite{cajets} with a distance parameter of $1.5$ (CA15 jets) are utilized to identify the Higgs boson candidate.  The large cone size is chosen in order to include events characterized by the presence of Higgs bosons with medium boost ($p_\mathrm{T}$ of the order of 200 GeV).
To reduce the impact of particles arising from pileup interactions,  the four-vector of each PF candidate is scaled with a weight calculated with the pileup per particle identification (PUPPI) algorithm ~\cite{puppi} prior to the clustering.
The absolute jet energy scale is corrected using calibrations derived from data~\cite{jec}. The CA15 jets are also required to be central ($\eta$ $<$ 2.4).
The ``soft-drop'' (SD) jet grooming algorithm~\cite{msd} is applied to remove soft wide-angle radiation from the jets. We refer to the groomed mass of the CA15 jet as $m_\text{SD}$. 

The ability to identify two b quarks inside a single CA15 jet is
crucial for this search. A likelihood for the CA15 jet to contain two
b quarks is derived by combining the information from primary and secondary vertices and tracks in a multivariate discriminant optimized to distinguish the Higgs to \bb decays from energetic quarks or gluons~\cite{Sirunyan:2017ezt} that appear inside the CA15 jet cone.
%The chosen working point of double b-tagger ($>$ 0.75) corresponds to the same background efficiency of the medium 
%working point of minimum of the CSV score of subjets. 
The working point chosen for this algorithm (the ``double-b tagger'')
corresponds to an identification efficiency of 50\% for a \bb system
with a \pt of 200 GeV, and a probability of about 10-13\% for
misidentifying CA15 jets originating from other combinations of quarks
or gluons. The efficiency of the algorithm increases with the \pt~of the \bb system.

%These numbers have been determined in simulation samples with at least one reconstructed CA15 jet with $\pt>200$\GeV and an otherwise inclusive selection. 

Energy correlation functions are used to identify the two-prong
structure in the CA15 jet expected from a Higgs boson decay to two b
quarks, and to distinguish it from QCD-like jets (i.e., jets that
do not originate from a heavy resonance decay) and jets from the hadronic decays of top quarks. The energy correlation functions are sensitive to correlations among the constituents
of CA15 jets (the PF candidates)~\cite{ecf}. They are $N$-point correlation functions ($e_N$) of the constituents' momenta, weighted by the angular separation of the constituents.
%Discriminating variables are constructed by using ratios of these functions.
%\begin{linenomath}
%\begin{equation}
%  \frac{_ae_N^\alpha}{(_be_M^\beta)^x}, \text{ where $M\leq N$ and $x= \frac{a\alpha}{b\beta}$}.
%  \label{eq:ecf}
%\end{equation}
%\end{linenomath}
%                                                                                                                                                                                                
%In Eq.~(\ref{eq:ecf}), the six free parameters are
%$N,a,\alpha,M,b,~\mathrm{and}~\beta$ and the value of $x$ is chosen to
%make the ratio dimensionless. An $N$-pronged jet is expected to have
%$e_N \gg e_M$ for $M>N$. 
As motivated in Ref.~\cite{ecf}, the ratio $N_2 =
e_3^{(\beta)}/(e_2^{(\beta)})^2$ is proposed as a two-prong tagger for
the identification of the CA15 jet containing the Higgs boson decay
products; the parameter $\beta$, which controls the weighting of the angles between constituent pairs in the computation of
the $N_2$ variable, is chosen to be 1 as the value that gives the best two-prong jet identification. 

%\begin{equation}
%  N_2(\beta) = \frac{e(2,3,\beta)}{e(1,2,\beta)^2}
%\end{equation}

%When computing the $N_2$ variable, only the first one hundred highest $p_T$ constituents that survive soft-drop grooming are used.

It is noted that requiring a jet to be two-pronged by using a jet substructure variable,
such as $N_2$, will affect the shape of the distribution of $m_\text{SD}$ for the
background processes. In this search, the value of $m_\text{SD}$ is required to be consistent with the Higgs boson mass.
It is therefore desirable to preserve a smoothly falling jet mass
distribution for QCD-like jets. 
As motivated in Ref.~\cite{ddt}, the stability of $N_2$ is tested against the variable
$\rho=\ln(m_{\text{SD}}^2/\pt^2)$, since the distribution of $\rho$ is expected to be stable in QCD-like jets.
%: since the jet mass distribution for QCD multijet
%events is expected to scale with \pt, decorrelating the $N_2$ variable
%as a function of $\rho$ and \pt would be the most appropriate procedure. 
The decorrelation strategy described in Ref.~\cite{ddt} is applied,
choosing a background efficiency of 20\%, which corresponds to a
signal efficiency of roughly 50\%. This results in a modified tagging
variable, which we denote as $N_2^\text{DDT}$, where the superscript DDT stands for designing decorrelated taggers~\cite{ddt}. Fig.~\ref{n2ddt} shows the expected distribution of the $N_2^\text{DDT}$ for CA15 jets matched to a Higgs boson decaying to a $\mathrm{b\bar{b}}$ pair, together with the distributions expected for CA15 jets matched to hadronically decaying top quarks and for QCD-like CA15 jets.
%This choice of background (or, equivalently, signal) efficiency has been optimized in order to maximize the $S/B$ ratio while retaining high enough event yield for the data-driven estimation of the background processes.
\begin{figure}
\centering
  \includegraphics[width=0.65\textwidth]{figures/ddt_N2DDT_ns.pdf} \\
\caption{The $N_2^\text{DDT}$ distribution as expected for CA15 jets matched to a Higgs boson decaying to a $\mathrm{b\bar{b}}$ pair (in red), compared to the expected distribution for CA15 matched to the decay products of top quarks decaying hadronically (in grey). The distribution corresponding to CA15 jets that do not originate from a heavy resonance decay is also shown in blue.}
\label{n2ddt}
\end{figure}


This search also utilizes narrow jets clustered
with the anti-$\kt$ algorithm~\cite{Cacciari:2008gp}, with a distance
parameter of $0.4$ (“AK4” jets). Narrow jets originating from b quarks are identified using the combined secondary vertex (CSVv2) algorithm \cite{Sirunyan:2017ezt}. The working point used in this search has a b jet identification efficiency of 81\%, a charm jet selection efficiency of 37\%, and a 9\% probability of misidentifying light-flavor jets~\cite{Sirunyan:2017ezt}. Jets that are b tagged are required to be central ($|\eta|<2.4$).
%%An event with any additional narrow jet passing the loose b tagging requirement is vetoed to reduce the number of events containing top quarks. 

Electron reconstruction requires the matching of a supercluster in the ECAL with a track in the silicon tracker.
Identification criteria~\cite{Khachatryan:2015hwa} based on the ECAL shower shape and the consistency of the track with the primary vertex are imposed. The reconstructed electron is required to be within $|\eta|< 2.5$, excluding the transition region $1.44<|\eta|<1.57$ between the ECAL barrel and endcap. Muons candidates are selected by two different reconstruction approaches~\cite{CMSMuonJINST}: the one in which tracks in the silicon tracker are matched to a track
 segment in the muon detector, and the other one in which a track
 fit spanning the silicon tracker and muon detector is performed
 starting with track segments in the muon detector. Candidates that are found by both the approaches are considered as single candidates.
Further identification criteria are imposed on muon candidates to reduce the number of misidentified hadrons and poorly measured mesons tagged as muons~\cite{CMSMuonJINST}. 
These additional criteria include requirements on the number of hits in the tracker and in the muon systems, the fit quality of the global muon track, and its consistency with the primary vertex.
Muon candidates with $|\eta|< 2.4$ are considered in this analysis. 
With electron and muon candidates, the minimum \pt is required to be 10\GeV. Isolation is required for both the objects.
Hadronically decaying $\tau$ leptons, $\tau_\mathrm{h}$, are reconstructed using the hadron-plus-strips
  algorithm~\cite{CMSTauJINST}, which uses the charged hadron and neutral electromagnetic objects 
to reconstruct intermediate resonances into which the $\tau$ lepton decays. The $\tau_\mathrm{h}$
candidates with $\pt>18\GeV$ and $|\eta|< 2.3$ are considered~\cite{Khachatryan:2015hwa,Chatrchyan:2013sba,CMSTauJINST}. Photon candidates, identified by means of requirements on the ECAL energy distribution and its distance to the closest track, must have $\pt>15\GeV$ and $|\eta|< 2.5$. %We veto both $\tau_\mathrm{h}$ candidates and photon candidates in the search presented. %Events containing an isolated electron, muon, or an isolated hadronic tau passing these criteria are 

The missing transverse momentum $\vec{p}_{\mathrm{T}}^{\mathrm{miss}}$ is defined as the negative vectorial sum of the \pt of all the reconstructed PF candidates. Its magnitude is denoted as \MET. Corrections to jet momenta are propagated to the \MET calculation as well as event filters \cite{CMS-PAS-JME-16-004} are used to remove spurious high \MET events caused by instrumental noise in the calorimeters or beam halo muons~\cite{CMS-PAS-JME-16-004}. The filters remove about 1\% of signal events.

%For the control regions involving lepton(s) in the final state, the hadronic recoil against the boson is computed by removing the \pt of the lepton(s) from the \MET computation. The recoil is given by: 
%\begin{equation} 
%\vec{U}  = \MET + p_{\mathrm{T}}^{\ell,\ell\ell} 
%\end{equation} 
%where the \pt$^{l}$ is \pt of lepton(s) in the W (Z) control regions. 
