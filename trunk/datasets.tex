%This analysis uses pp collision data at $\sqrt{s}=13\TeV$ recorded by the CMS experiment at the LHC during 2016.
%
%The data correspond to an integrated luminosity of $35.9\,\text{fb}^{-1}$. 
%
%Control regions obtained by requiring the presence of muons or electrons are employed in the search for $\MET+\mathrm{h}(\mathrm{b}\bar{\mathrm{b}})$; therefore, the \MET and SingleElectron datasets are used for offline analysis. 
%
%The \MET dataset includes muon pass-through triggers, which allow for selecting events with high $\MET$ or high muon recoil.

\section{The CMS detector}

The CMS detector, described in detail in Ref.~\cite{CMSdetector}, is a multipurpose apparatus designed to study high-transverse momentum  processes in proton-proton and heavy ion collisions.
%                                                                                                                                                                                                    
A superconducting solenoid occupies its central region, providing a magnetic field of 3.8\unit{T} parallel to the beam direction.
%                                                                                                                                                                                                    
Charged particle trajectories are measured using silicon pixel and strip trackers that cover a pseudorapidity region of $\abs{\eta} < 2.5$.
%                                                                                                                                                                                                    
A lead tungstate (PbWO$_4$) crystal electromagnetic calorimeter (ECAL) and a brass and scintillator hadron calorimeter  surround the tracking volume and extend to $\abs{\eta} < 3$.
%                                                                                                                                                                                                    
The steel and quartz-fiber forward Cherenkov hadron calorimeter extends the coverage to $\abs{\eta} < 5$.
%                                                                                                                                                                                                    
The muon system consists of gas-ionization detectors embedded in the steel flux-return yoke outside the solenoid and covers $\abs{\eta} < 2.4$.
%                                                                                                                                                                                                    
The return yoke carries a 2\unit{T} return field from the solenoid.
%                                                                                                                                                                                                    
Online event selection is accomplished via the two-tiered CMS trigger
system. The first level is designed to select events in less than
4\mus, using information from the calorimeters and muon detectors.
%                                                                                                                                                                                                    
Subsequently, the high-level trigger processor farm reduces the event rate to several hundred Hz.


\section{Simulation of data samples}

The signal processes are simulated at leading order (LO) accuracy in quantum chromodynamics (QCD) perturbation theory using the \MADGRAPH{\textsc 5\_aMC@NLO} v2.4.2~\cite{amcatnlo} program.
%
To model the contributions from SM Higgs boson processes as well as
from the \ttbar and single top quark backgrounds, we use the {\sc Powheg~v2}~\cite{Nason:2004rx,Frixione:2007vw,Alioli:2010xd} generator. These processes are generated at next-to-leading order (NLO) in QCD. The \ttbar production 
cross section is further corrected using calculations at 
next-to-next-to-leading (NNLO) order in QCD including NNLO logarithmic corrections for soft gluon radiation~\cite{ttbarNNLO}. 
% 
Events with multiple jets produced through the strong interaction (referred to as QCD multijet events) are generated at LO using \MADGRAPH{\textsc 5\_aMC@NLO} v2.2.2 with up to four partons. The MLM prescription~\cite{mlm} is used for matching these partons to parton shower jets.
%
Simulated samples of Z+jets and W+jets processes are generated at LO using \MADGRAPH{\textsc 5\_aMC@NLO} v2.3.3. Up to four additional partons are considered in the matrix element and matched to their parton showers using the MLM technique.
%
The V+jets samples are corrected by weighting the \pt of the respective boson with NLO QCD corrections obtained from large samples of events generated with \MADGRAPH{\textsc 5\_aMC@NLO} and the FxFx merging technique~\cite{fxfx} with up to two additional jets stemming from the matrix element calculation.
%
These samples are further corrected by applying NLO electroweak corrections obtained from calculations~\cite{Kuhn:2005gv,Kallweit:2015fta,Kallweit:2015dum} that depend on the boson \pt.
%
Predictions for the SM diboson production modes WW, WZ, and ZZ are obtained at LO with the {\sc \PYTHIA 8.205}~\cite{Sjostrand:2014zea} generator and normalized with NLO corrections from \MCFM~\cite{MCFM}. 
%


The LO or NLO NNPDF 3.0 parton distribution functions (PDFs)~\cite{Ball:2014uwa} are 
used, depending on the QCD order of the generator used for each physics 
process. 
%
Parton showering, fragmentation, and hadronization are simulated with {\sc \PYTHIA 8.212} using the CUETP8M1 tune~\cite{ue1,ue2}. 
%
Interactions of the resulting final state particles with the CMS detector are simulated using the \GEANTfour program~\cite{geant4}.
%
Additional inelastic proton-proton interactions in the same or a neighboring bunch crossing (pileup) are included in the simulation.
%
The pileup distribution is adjusted to match the corresponding distribution observed in data. 
