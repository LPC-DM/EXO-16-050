.\section{Systematic uncertainties}

 Nuisance parameters are introduced into the likelihood fit to represent the systematic uncertainties of the search. They can either affect the rate or  the shape of \ptmiss ($U$) for a given process in the SR (CRs) and can be constrained in the fit. Shape uncertainties are incorporated by means of a prior Gaussian distribution, while rate uncertainties are given a prior log-normal distribution. A summary of systematic uncertainties is presented in Table~\ref{tab:systs}.


\begin{table}\footnotesize
\begin{center}
  \caption{Systematic uncertainties, along with the type (rate/shape) of uncertainty and the affected processes. For the rate uncertainties, the percentage effect on the rate is quoted.}
\begin{tabular}{l r r}
  \hline\hline
Systematic uncertainty & Type & Processes \\
\hline
double-b tagging & rate/shape & Z+jets, W+jets, \ttbar, SM h, signal\\
heavy flavor fraction & 4, 5\% & W+jets, Z+jets, respectively\\
\ptmiss~trigger efficiency & 1\% & all \\
single-electron trigger & 1\% & all \\
lepton efficiency & 1\% per lepton & all \\
$\tau$ lepton veto & 3\% & all \\
%AK4 b tagging & shape & all \\
\ptmiss~trigger muon multiplicity & shape & \ttbar, Z+jets, W+jets \\
theoretical cross section & 20\% & t, diboson\\
multijet normalization & 100\% & multijet \\
\ptmiss magnitude & 5\% & all \\
CA15 jet energy & 4\% & t, diboson, multijet, SM h, signal \\
$N_2^\text{DDT}$ efficiency & 7\% & diboson, SM h, signal \\
luminosity & 2.5\% & t, diboson, multijet, SM h, signal \\
bin-by-bin statistics & shape & \ttbar, Z+jets, W+jets \\
\hline\hline
  \end{tabular}
\label{tab:systs}
\end{center}
\end{table}

Scale factors are used to correct for differences in the double-b tagger misidentification efficiencies between data and prediction from simulation for W/Z+jets production and for \ttbar production. These scale factors affect only the overall rates of the respective processes, and they are measured directly in the fit by simultaneously fitting events that pass or fail the double-b tag requirement. The uncertainties on the scale factor measurements are the one with the largest impact on the measurement. To take into account the variation of the double-b tagging efficiency
introduced by the uncertainty in the heavy flavor fraction of W/Z+jets
events, the efficiencies are reevaluated after varying the heavy
flavor component in the simulation. The difference in the efficiency with respect to the nominal efficiency value is taken as a systematic uncertainty.
The correlation between the double-b tagger and the \ptmiss (or $U$) introduces a shape uncertainty in the transfer factors from the CRs with events that satisfy the inverted double-b tag criterion. This shape uncertainty is anchored at the first bin where the magnitude of the uncertainty is 0\%, and increases to 8\% in the last hadronic recoil bin. These numbers are extrapolated reflecting the level of correlation between the tagger and the \ptmiss/$U$, estimated by fitting the profile of the two-dimensional distribution double-b-vs-\ptmiss/$U$.
For the signal and the SM h processes, an uncertainty in the double-b tagging efficiency is applied that is 3\% for a CA15 jet with $\pt<350\GeV$, 4\% for the intermediate \pt~regime, and 8\% for $\pt>800$\GeV. These numbers have been derived through a measurement performed in a sample enriched in multijet events with double-muon-tagged $\text{g}\to\text{b}\bar{\text{b}}$ splittings. 

An uncertainty of $21\%$ in the heavy flavor fraction of
W+jets is reported in previous CMS measurements~\cite{Khachatryan:2014uva,Chatrchyan:2013uza}. Similarly, the uncertainty on the Z+heavy flavor fraction is measured to be $22\%$~\cite{Khachatryan:2014zya,Chatrchyan:2014dha}. Depending on the b tagging requirement, the uncertainty on the heavy flavor fraction on such processes determines an uncertainty on the overall rate (the sum of heavy and light flavor components) for both W and Z+jets. To account for this effect, a flat $4\%$ is assigned to the W+jets process, while a flat $5\%$ is assigned to the Z+jets process. They are included as rate uncertainties, correlated across all regions defined by the same b tagging requirement and anti-correleted across all regions defined by inverting it.

The \ptmiss trigger efficiency is parametrized as a function of $p_\text{T,trig}^\text{miss}$. The uncertainty in its measurement is a flat 1\% and is included in the fit as a rate uncertainty.
The efficiencies of the single-electron triggers are parametrized as a
function of the electron \pt and an associated flat 1\% systematic uncertainty is added into the fit.

Uncertainties in the selection efficiencies per selected muon or electron amount to $1\%$, correlated across all $U$ bins.

Correlated across all $U$ bins is also the uncertainty in the $\tau$ lepton veto, that amounts to a flat $3\%$.

The transfer factors are affected by uncertainties in the efficiencies
of lepton identification.  For example, differences on the order of
2\% in \MET trigger efficiencies are observed between single-muon and
dimuon events at lower $U$ values and represent an additional
systematic uncertainty in the transfer factors for those processes whose prediction relies on data events in the single-muon and
dimuon CRs (\ttbar, W+jets, and Z+jets production). As these uncertainties depend on the momentum of the identified lepton they can change the shape of the $U$ distribution and are thus treated as shape uncertainties.

A systematic uncertainty of $20\%$ is assigned to the single top quark background yields~\cite{Chatrchyan:1642680} and is correlated between the SR and the CRs. An uncertainty of $20\%$ is also assigned to the diboson production cross section~\cite{Khachatryan:2016txa,Khachatryan:2016tgp} and correlated across the SR and CRs.
%

Being a negligible background source, a conservative uncertainty of $100\%$ is assigned to the QCD multijet yield. This uncertainty is estimated using a sample enriched in multijet events. The sample is obtained by vetoing leptons and photons, by requiring $\ptmiss>250$\GeV, and that the minimum azimuthal angle between $\vec{p}_{\mathrm{T}}^{\mathrm{miss}}$ and the jet directions to be less than $0.1$\,rad. One nuisance parameter represents the uncertainty in QCD multijet yields in the signal region, while separate nuisance parameters are introduced for the muon CRs and for electron CRs.

A flat 5\% rate uncertainty in \ptmiss is obtained following the standard CMS method~\cite{Khachatryan:2014gga} and is assigned to each processes estimated from simulation.

Similarly, a flat 4\% rate uncertainty due to the imperfect knowledge of the CA15 jet energy scale~\cite{jec} is assiged to the backgrounds obtained from simulation.

A 7\% rate uncertainty in the efficiency of the the substructure variable
$N_2^\text{DDT}$ to identify two-prong CA15 jets is assigned to all processes where the decay of a resonance inside the CA15 jet cone is expected. Such processes include signal production together with SM h and diboson production.  
The uncertainty has been derived from the efficiency measurement obtained by performing a fit in a control sample
enriched in semi-leptonic \ttbar events, where the CA15 jet originates
from the W boson that comes from the hadronically decaying top quark. 
%

A rate uncertainty of 2.5\% in the luminosity measurement~\cite{CMS-PAS-LUM-17-001} is included and assigned to processes determined from simulation. In these cases, appropriate QCD scale and PDF uncertainties are also applied.
%

A shape uncertainty is considered for all the processes predicted with data in CRs obtained by introducingNuisance parameters corresponding to bin-by-bin statistical uncertainties on the transfer factors. 
%

Due to the similarity of selection requirements between the SR and CRs, uncertainties in the higher order NLO or EW corrections, or in the PDF, cancel out when building the ratio of yields predicted in the SR and the corresponding CR.

%Unlike the uncertainties described above, uncertainties on the signal predictions from QCD scale and PDF variations are not propagated as nuisance parameters. Instead, they are treated as uncertainties in the inclusive signal cross section. Due to the imperfect knowledge of PDFs at higher $x$, where $x$ is the momentum fraction carried by the partons participating in the hard interaction, the uncertainties increase with the mass of the final state and range from 4\% to 20\%.

