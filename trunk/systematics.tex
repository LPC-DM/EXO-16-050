\section{Systematic uncertainties}

 Nuisance parameters are introduced into the likelihood fit to represent the systematic uncertainties of the search. They can either affect the rate or  the shape of \ptmiss ($U$) for a given process in the SR (CRs) and can be constrained in the fit. Shape uncertainties are incorporated by means of a Gaussian prior distribution, while rate uncertainties are given a log-normal prior distribution.

The \MET trigger efficiency is parametrized as a function of the
\MET. The uncertainty in its measurement is about 1\% and is included in the fit as a rate uncertainty.
The efficiencies of the single-electron triggers are parametrized as a
function of the electron \pt and an associated flat 1\% systematic uncertainty is added into the fit.

Uncertainties in the selection efficiencies amount to $1\%$ per selected muon or electron, while the uncertainty in the $\tau$ lepton veto is $3\%$; these rate uncertainties are correlated across all $U$ bins.

An uncertainty of $21\%$ in the heavy flavor fraction of
W+jets is reported in CMS measurements of inclusive W+jets
\cite{Khachatryan:2014uva} and W+heavy flavor
\cite{Khachatryan:2014uva,Chatrchyan:2013uza} production. Similarly, the uncertainty on the Z+heavy flavor fraction is measured to be $22\%$~\cite{Khachatryan:2014zya,Chatrchyan:2014dha}.
The magnitudes of the uncertainties in the W$/$Z+jets yields due to varying the heavy flavor components are different for each region (depending on the b tagging requirements) and range from $4\%$ to $5\%$ of the central W$/$Z+jets prediction. They are included as rate uncertainties and are correlated across all regions, but not correlated between the W+jets and Z+jets processes.
%
%These W+heavy flavor fraction uncertainties are correlated between all regions in the fit. 
%

The transfer factors $T$ are affected by uncertainties in the efficiencies of b-tagging or mistagging narrow isolated jets and by uncertainties related to lepton identification.  Differences on the order of 2\% in \MET trigger efficiencies observed between single-muon and dimuon events at lower $U$ values represent an additional systematic uncertainty in the transfer factors of the single-muon and dimuon CRs. All of these uncertainties can change the shape of the $U$ distribution. Shape uncertainties  due to higher order NLO or EW corrections, or PDF uncertainties, cancel out in the transfer factors when building the ratio of yields predicted in the SR and the corresponding CR due to the similarity of selection requirements between the SR and CRs.


\begin{table}\footnotesize
\begin{center}
  \caption{Systematic uncertainties, along with the type (rate/shape) of uncertainty and the affected processes. For the rate uncertainties, the percentage effect on the rate is quoted.}
\begin{tabular}{l r r}
  \hline\hline
Systematic uncertainty & Type & Processes \\
\hline
AK4 b tagging & shape & all \\
double-b tagging & shape & Z+jets, W+jets, \ttbar, SM h, signal\\
\ptmiss~trigger muon multiplicity & shape & Z+jets, W+jets \\
QCD scales & shape & SM h \\
PDF & shape & SM h \\
\ptmiss magnitude & 5\% & all \\
\ptmiss~trigger efficiency & 1\% & all \\
single-electron trigger & 1\% & all \\
lepton efficiency & 1\% per leg & all \\
$\tau$ lepton veto & 3\% & all \\
luminosity & 2.5\% & t, diboson, multijet, SM h, signal \\
CA15 jet energy & 4\% & t, diboson, multijet, SM h, signal \\
$N_2^\text{DDT}$ efficiency & 7\% & diboson, SM h, signal \\
theoretical cross section & 20\% & t, diboson\\
heavy flavor fraction & 4-5\% & Z+jets, W+jets\\
multijet normalization & 100\% & multijet \\
\hline\hline
  \end{tabular}
\label{tab:systs}
\end{center}
\end{table}





Two types of scale factors are used to correct for a potential difference in the double-b tagger misidentification efficiencies between data and prediction, one for W/Z+jets production and another for \ttbar production. Both factors affect only the overall rates of the respective processes, and they are measured directly in the fit by simultaneously fitting events that pass or fail the double-b tag requirement. The scale factors for W/Z+jets production are included in the fit as an unconstrained nuisance parameter. 
%
To take into account the variation of the double-b tagging efficiency
introduced by the uncertainty in the heavy flavor fraction of W/Z+jets
events, the efficiencies are reevaluated after varying the heavy
flavor component in the Monte Carlo simulation. The difference in the efficiency with respect to the nominal efficiency value is taken as a systematic uncertainty.
%
The values of the scale factor for \ttbar and its uncertainty are taken from an independent measurement in a statistically independent data sample. %In this case the efficiency is fixed, since no fluctuation is introduced by any effect like the varying heavy flavor fraction for W/Z+jets. 

%
%For W/Z+jets production, scale factors to correct the double-b tagger mis-identification efficiency are measured directly in the fit. This in-situ calibration is accomplished by performing the fit simultaneously with events that fail the double-b tagger requirement. The scale factor is included in the fit as an unconstrained nuisance. 
%
%To take into account the variation of the double-b tagger efficiency introduced by the uncertainty in the heavy flavor fraction of W/Z+jets events, up and down variation efficiencies are estimated. The difference between the up and down efficiency with respect to the central value is taken as a systematic uncertainty. 
%
%Additionally, a $U$-dependent shape uncertainty is put on the transfer factors tying the SR with the CRs that have an inverted requirement on the double-b tagger output; the uncertainties are 0\%, 2\%, 4\%, and 8\% for the four $U$ bins. It should be noted that the first bin is still allowed to vary due to the aforementioned scale factor that corrects the double-b pass/fail ratio globally.
%
%A similar approach is used for the measurement of the scale factor that corrects the mis-identification probability of \ttbar events. In this case the efficiency is fixed, since no fluctuation is introduced by any effect like the varying heavy flavor fraction for W/Z+jets.
%
In addition, a shape uncertainty that grows with \ptmiss or $U$ is applied on the transfer factors constraining the W/Z+jets backgrounds from the CRs with events that satisfy the inverted double-b tag criterion. This is done to account for potential effects with a residual \pt dependence. The shape uncertainty is anchored at the first bin where the magnitude of the uncertainty is 0\%, and increases to 8\% in the last hadronic recoil bin. 
%

A systematic uncertainty of $20\%$ is assigned to the single top quark background yields~\cite{Chatrchyan:1642680} and is correlated between the SR and the CRs. 
%
An uncertainty of $20\%$ is also assigned to the diboson production cross section~\cite{Khachatryan:2016txa,Khachatryan:2016tgp} and correlated across the SR and CRs.
%

Although an insignificant background, a conservative uncertainty of $100\%$ is used for the QCD multijet yield. 
%
This uncertainty is estimated using a sample enriched in multijet events. The sample is obtained by vetoing leptons and photons and by requiring $\ptmiss>250$\GeV and the minimum azimuthal angle between $\vec{p}_{\mathrm{T}}^{\mathrm{miss}}$ and the jet directions to be less than $0.1$\,rad.  The uncertainty is correlated between regions with the same
source of the faked object. That is, one nuisance parameter represents the
uncertainty in QCD multijet yields in the signal region and separate
nuisance parameters are introduced for the muon CRs and for electron CRs.
%

For processes estimated from Monte Carlo simulation, uncertainties in \ptmiss are obtained directly from simulation and propagated to $U$ following the standard CMS method~\cite{Khachatryan:2014gga}, which includes the jet energy correction uncertainties applied to the jets and \MET\ \cite{jec}. This results in a 4\% rate uncertainty in the backgrounds obtained from simulation due to the imperfect knowledge of the CA15 jet energy scale.
%
The \MET uncertainty amounts to a 5\% rate uncertainty that is included on each process in the final fit.
%

All processes for which a two-prong CA15 jet stemming from a resonance
decay is expected (the signal process, SM h production, diboson
production) contribute a 7\% rate uncertainty that is correlated
across all analysis regions. This corresponds to the uncertainty in
passing or failing the requirement on the substructure variable
$N_2^\text{DDT}$. The uncertainty has been derived in a control sample
enriched in \ttbar events, where the CA15 jet most often originates
from a hadronically decaying W boson that comes from the decay of one of the top quarks, but the corresponding b jet is out-of-cone. 
%
For the signal and the SM h processes, an uncertainty in the double-b tagging efficiency is applied that is 3\% for a CA15 jet with $\pt<350\GeV$, 4\% for the intermediate \pt~regime, and 8\% for $\pt>800$\GeV. These numbers have been derived through a measurement performed in a sample enriched in multijet events with double-muon-tagged $\text{g}\to\text{b}\bar{\text{b}}$ splittings. 
%

A systematic uncertainty of 2.5\% in the luminosity measurement~\cite{CMS-PAS-LUM-17-001} is included whenever the yields for a process in a specific bin are not determined from data. In these cases, appropriate QCD scale and PDF uncertainties are applied, too.
%

The impact of statistical uncertainties in predicted yields from simulation-driven backgrounds is negligible. However,  additional nuisance parameters corresponding to bin-by-bin statistical uncertainties on the transfer factors $T$ are considered. 
%

A summary of systematic uncertainties is presented in Table~\ref{tab:systs}.

Unlike the uncertainties described above, uncertainties on the signal predictions from QCD scale and PDF variations are not propagated as nuisance parameters. Instead, they are treated as uncertainties in the inclusive signal cross section. Due to the imperfect knowledge of PDFs at higher $x$, where $x$ is the momentum fraction carried by the partons participating in the hard interaction, the uncertainties increase with the mass of the final state and range from 4\% to 20\%.

