

Table \ref{tab:eventYieldTable} shows, for the \Hbb channel, the 
SR post-fit yields for each background and signal mass point along with the 
sum of the statistical and systematic uncertainties for the resolved and boosted regimes. 

For the \HGG channel, when applying the event selection to the data, two events are observed in the $m_{\gamma\gamma}$ sidebands and are used to 
evaluate the magnitude of the nonresonant background as described in Section~\ref{sec:bkg_model}. 
This yields an expected number of $0.38 \pm 0.27 (stat)$ nonresonant background events in the SR.
Expected resonant background contributions are taken from the simulation as detailed in Section~\ref{sec:bkg_model} and are $0.057 \pm 0.006 (stat)$ events considering both the Vh production (dominant) and the gluon fusion mode. Zero events are observed in the SR in the data.



Since no excess of events has been observed over the SM background expectation in the signal region, the results of this search are interpreted in terms of an upper limit on the production cross 
section of DM candidates in association with a Higgs boson via $\PZpr \rightarrow \Az \Ph \rightarrow \chi \bar{\chi} \PAQb\PQb(\gamma\gamma)$. 
The upper limits are computed at 95\% confidence level (CL) using a modified frequentist method (CL$_s$) \cite{yellowReport, bib:CLS1, bib:CLS2} computed with an asymptotic approximation \cite{bib:CLS3}. 
A profile likelihood ratio is used as the test statistic in which systematic uncertainties are modeled as nuisance parameters.
These limits are obtained as a function of \mzp and \maz for both Higgs boson decay channels and for the combination of the two. The two decay channels are combined using the branching ratios predicted by the SM.
In the combination of the two analyses, all signal and \MET-related systematic uncertainties as well as the systematic uncertainty on the integrated luminosity 
are assumed to be fully correlated.

Figure~\ref{fig:limitsexpected} (left) shows the 95\% CL expected and observed limits on the dark matter production cross section for 
\Hbb and \HGG for \maz = 300 GeV. 
Figures~\ref{fig:limitsexpected} (right) and~\ref{fig:limit2d} show the 95\% CL  expected and observed upper limits on the signal strength. 
For \maz = 300 GeV, the \mzp mass range 600 to 1780 \GeV is expected to be excluded with a 95\% CL when the signal model 
cross section is calculated using \gzp = 0.8. 
The observed data excludes, for \maz = 300 \GeV the \zp mass range of 600 to 1860 GeV. 
When the signal model cross section is calculated using the constrained \gzp, the expected exclusion range is 830 to 1890 GeV, 
and with the observed data the exclusion range is 770 to 2040 GeV. 
