\section{Data and simulated samples}\label{sec:datasets}
%This analysis uses pp collision data at $\sqrt{s}=13\TeV$ recorded by the CMS experiment at the LHC during 2016.
%
%The data correspond to an integrated luminosity of $35.9\,\text{fb}^{-1}$. 
%
%Control regions obtained by requiring the presence of muons or electrons are employed in the search for $\MET+\mathrm{h}(\mathrm{b}\bar{\mathrm{b}})$; therefore, the \MET and SingleElectron datasets are used for offline analysis. 
%
%The \MET dataset includes muon pass-through triggers, which allow for selecting events with high $\MET$ or high muon recoil.


Signal samples are generated at leading order (LO) accuracy in quantum chromodynamics perturbation theory (QCD) using the \MADGRAPH{\textsc 5\_aMC@NLO}~\cite{amcatnlo} program.
%


To model the expectation from SM Higgs boson backgrounds as well as the \ttbar and single top quark backgrounds, the {\sc Powheg~v2}~\cite{Nason:2004rx,Frixione:2007vw,Alioli:2010xd} generator is used. All these processes are generated at the next-to-leading order (NLO) in QCD.
% 
Events with multiple jets produced through the strong interaction (referred to as QCD multijet events) are generated at LO using \MADGRAPH{\textsc 5\_aMC@NLO} v2.3.3.
%
Simulated samples of Z+jets and W+jets processes are generated at LO using \MADGRAPH{\textsc 5\_aMC@NLO} v2.3.3. Jets stemming from the matrix element calculations are matched to parton shower jets using the MLM prescription~\cite{mlm}.
%
The samples are corrected by weighting the \pt of the respective boson with NLO QCD corrections obtained from large samples of events generated with \MADGRAPH{\textsc 5\_aMC@NLO} and the FxFx merging technique~\cite{fxfx}.
%
The samples are further corrected by applying NLO electroweak corrections obtained from calculations~\cite{Kuhn:2005gv,Kallweit:2015fta,Kallweit:2015dum} that depend on boson \pt.
%
Predictions for the associated production of SM vector boson (i.e., diboson) production are obtained at LO with the {\sc Pythia 8.205}~\cite{Sjostrand:2014zea} generator.
%


For all the processes the LO or NLO the NNPDF30 parton distribution function~\cite{Ball:2014uwa} is used at LO or NLO. 
%
Parton showering, fragmentation, and hadronization are simulated with {\sc Pythia 8.2} using the CUETP8M1~\cite{ue1,ue2} tune. 
%
A {\sc Geant4}-based simulation of the CMS detector~\cite{geant4} is applied to all the simulated processes. 
%
Additional inelastic proton-proton interactions in the same or a neighboring bunch crossing (pileup) is included in the simulation.
%
The pileup distribution is corrected to match the corresponding distribution observed in data. 
