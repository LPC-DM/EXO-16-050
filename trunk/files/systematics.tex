\section{Systematic uncertainties}\label{sec:systematics}

Systematic uncertainties are treated as constrained nuisance
parameters in the fit.

The \MET trigger efficiency is parametrized as a function of the \MET. The uncertainty on its measurement is about 1\% and it is included in the fit as the systematic uncertainty.
The efficiencies of the single electron triggers are measured in data and parametrized as a function of the electron \pt. A 1\% systematic uncertainty on this measurement is added in the final fit.
%
Uncertainties in the selection efficiencies amount to $1\%$ per selected muon or electron, while the uncertainty in the tau lepton veto is $3\%$; these are correlated across all $U$ bins.

An uncertainty of $21\%$ on the heavy flavor fraction of
W+jets using CMS measurements of inclusive W+jets
\cite{Khachatryan:2014uva} and W+heavy flavor
\cite{Khachatryan:2014uva,Chatrchyan:2013uza} production is included. This
uncertainty is propagated to each of the SR and CRs by
varying the heavy flavor fraction up and down in the Monte Carlo
prediction, and is correlated between all regions. 
%
%These W+heavy flavor fraction uncertainties are correlated between all regions in the fit. 
%
A similar method is used for the Z+heavy flavor fraction uncertainty ($22\%$) using measurements of Z+jets production at CMS \cite{Khachatryan:2014zya,Chatrchyan:2014dha}. 
%
The resulting Z+heavy flavor uncertainty is also correlated between all regions, but is uncorrelated with the W+heavy flavor uncertainty. 
%
The magnitudes of these W$/$Z+heavy flavor uncertainties are different for each region (depending on the b-tagging requirements) and range from $4\%$ to $5\%$ of the nominal W$/$Z+jets prediction.
%


The transfer factors $R$ are afflicted with uncertainties in the efficiencies of b- or mis-tagging narrow isolated jets and with uncertainties related to lepton identification.  Differences in \MET trigger efficiencies observed between single-muon and dimuon events represent an additional systematic uncertainty on the transfer factors of the single-muon and dimuon CRs that is treated as a constrained nuisance parameter. Uncertainties on the shape of $U$ due to higher order NLO or EW corrections or PDF uncertainties cancel out in the transfer factors when building the ratio between the SR and the corresponding CR due to the similarity of selection requirements in the SR and CRs.

%
For W/Z+jets production, scale factors to correct the double-b tagger
mis-identification efficiency are measured with the signal extraction
directly in the final fit. This {\it in-situ} calibration is
accomplished by performing a simultaneous fit with events that fail the double-b tagger requirement. The scale factor is included in the fit as an unconstrained nuisance.
%
To take into account the variation of the double-b tagger efficiency introduced by the uncertainty on the heavy flavor fraction of W/Z+jets events, up and down variation efficiencies are estimated. The difference between the up and down efficiency with respect to the central value is taken as a systematic uncertainty. 
%
A similar approach is used for the the measurement of the scale factor that corrects the mis-identification probability of \ttbar events, where an independent measurement performed in an orthogonal data sample is used as prior for its central value and the measured uncertainty as its prior uncertainty. In this case the efficiency is fixed, since no fluctuation is introduced by any effect like the varying heavy flavor fraction for W/Z+jets.
%
In addition to the scale factors mentioned previously, which globally affect the yields per process and region, a shape uncertainty that grows with recoil is applied on the transfer factors from the CRs composed by events that do not satisfy the double-b requirement to the SR to cover effects with a potential residual \pt dependence. The shape uncertainty is anchored at the first bin where the magnitude of the uncertainty is 0\%, and it grows up to 8\% for the last hadronic recoil bin.
 
%
A systematic uncertainty of $20\%$ is ascribed to the single top quark background prediction~\cite{Chatrchyan:1642680} and is correlated between the signal region and the CRs. 
%
An uncertainty of $20\%$ is also assigned to the diboson production cross section~\cite{Khachatryan:2016txa,Khachatryan:2016tgp} and correlated across all channels. 
%

An uncertainty of $100\%$ is used for the overall QCD multijet yield. 
%
This is estimated using a sample enriched in QCD multijet events, obtained by requiring the minimum azimuthal angle between $\vec{p}_{\mathrm{T}}^{\mathrm{miss}}$ and the jet directions to be less than $0.1$. 
%

For processes estimated from Monte Carlo simulation, \MET uncertainties are obtained directly from simulation and propagated to $U$ following the standard CMS method~\cite{Khachatryan:2014gga}, which includes the jet energy correction uncertainties applied to the jets and the \MET\ ~\cite{jec}. 
%
The \MET uncertainty amounts to 5\% and is included as a rate uncertainty in the final fit.
%

A systematic uncertainty of 2.5\% in the luminosity measurement~\cite{CMS-PAS-LUM-17-001} is included for all the processes that are estimated using Monte Carlo simulation.
%

