\section{Introduction}

Dark matter (DM) is one of the most compelling pieces of evidence for physics beyond the standard model (SM)~\cite{FNAL_Review}. Cosmological observations demonstrate that around 85\% of the mass of the Universe is comprised of DM. These observations make it highly likely that DM is composed primarily of weakly interacting massive particles (WIMPs). If non-gravitational interactions exist between DM and SM particles, DM could be produced by colliding SM particles at high energy. In many theories, the pair production of DM particles in hadron collisions proceeds through a spin-0 or spin-1 bosonic mediator, with the DM particles leaving the detector without a measurable signature. One way to observe them is when they are produced in association with a visible SM particle, which could be emitted directly from a quark as initial state radiation (ISR) or as part of new effective vertex couplings of DM to SM particles. 
Unlike that of Ws, Zs, jets, or photons, the ISR of the SM Higgs boson~\cite{HiggsObs_ATLAS, HiggsObs_CMS, HiggsObs_CMS_Long} is highly suppressed due to the Yukawa-like nature of its coupling strenght.

However, the associated production of a Higgs boson and DM particles, called the mono-H final state, can occur in a scheme where the Higgs boson is part of the interaction producing the DM particles, directly probing the structure of the effective DM-SM coupling.

We present a search for DM production in association with Higgs boson
decaying into a pair of bottom quark. As the H$\rightarrow \mathrm{b
  \bar{b}}$ decay channel has the largest branching ratio of any SM
Higgs boson decays it provides the largest potential signal yield. Results are presented for the full dataset of approximately 35.9~\fbinv collected by the CMS experiment at the CERN LHC at a center of mass energy of 13\TeV~in 2016. 

