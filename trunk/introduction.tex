\section{Introduction} \label{intro}

Astrophysical evidence for dark matter (DM) is one of the most compelling motivations for
physics beyond the standard model
(SM)~\cite{dm1,dm2,dm3}. Cosmological observations demonstrate that
around 85\% of the matter in the universe is comprised of DM
\cite{planck} and are consistent with the hypothesis that DM is primarily composed of
weakly interacting massive particles (WIMPs). If non-gravitational
interactions exist between DM and SM particles, DM could be produced
by colliding SM particles at high energy. Assuming the pair
production of DM particles in hadron collisions happens through a
spin-0 or spin-1 bosonic mediator coupling to the initial state particles, the DM particles leave the
detector without a measurable signature. If DM particles are produced in association with a detectable SM particle, which could be emitted
as initial state radiation (ISR) from the interacting constituents of the colliding protons, or through
new effective vertex couplings between DM and SM particles, their
existence could be inferred via a large transverse momentum imbalance in the collision event. 


%Unlike in the case of W bosons, Z bosons, hadronic jets, or photons, 
While ISR production of the SM Higgs boson ($\Ph$)~\cite{HiggsObs_ATLAS, HiggsObs_CMS, HiggsObs_CMS_Long} is highly suppressed due to the Yukawa-like nature of its coupling strength, 
the associated production of a Higgs boson and DM particles
%,referred to as ``mono-h'' final state, 
can occur if the
Higgs boson is part of the interaction producing the DM particles~\cite{monoHiggs3,2HDM,PhysRevD.89.075017}.
Such a production mechanism allows one to directly probe the structure of the effective DM-SM coupling.

In this article, we  present a search for DM production in association
with a SM Higgs boson that decays into a pair of bottom quarks. As the 
$\Ph\rightarrow \bb$ decay mode has the largest branching ratio of all decay modes allowed in the SM, it provides the largest signal yield. The search is performed using the full data set collected by the CMS experiment~\cite{CMSdetector} at the CERN LHC at a center-of-mass energy of 13\TeV~in 2016, corresponding to an integrated luminosity of approximately 35.9\fbinv. 

Signal events are characterized by a large imbalance in the transverse momentum(or hadronic recoil), that indicates the presence of invisible particles, and by hadronic activity consistent with the production of a SM Higgs that decays into a b quark pair. Requirements on the mass of the reconstructed Higgs boson candidate, together with the identification of the products of hadronization of the two b quarks produced in the Higgs boson decay, define a data sample that is expected to be enriched in signal events. Several different SM processes can mimic such topology. Data in statistically independent control samples are used to predict the hadronic recoil distribution for these SM processes that constitute the largest source of background.
Both signal and background contributions to the data are extracted with a likelihood fit to the hadronic recoil distribution, performed simultaneously in all the different analysis subsamples.
