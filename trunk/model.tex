\section{Introduction} \label{intro}

Astrophysical evidence for dark matter (DM) is one of the most compelling motivations for
physics beyond the standard model
(SM)~\cite{dm1,dm2,dm3}. Cosmological observations demonstrate that
around 85\% of the matter in the universe is comprised of DM
\cite{planck} and are largely consistent with the hypothesis that DM is primarily composed of
weakly interacting massive particles (WIMPs). If nongravitational
interactions exist between DM and SM particles, DM could be produced
by colliding SM particles at high energy. Assuming the pair
production of DM particles in hadron collisions happens through a
spin-0 or spin-1 bosonic mediator coupled to the initial-state particles, the DM particles leave the
detector without a measurable signature. If DM particles are produced in association with a detectable SM particle, which could be emitted
as initial-state radiation (ISR) from the interacting constituents of the colliding protons, or through
new effective couplings between DM and SM particles, their
existence could be inferred via a large transverse momentum imbalance in the collision event. 


%Unlike in the case of W bosons, Z bosons, hadronic jets, or photons, 
While ISR production of the SM Higgs boson ($\Ph$)~\cite{HiggsObs_ATLAS, HiggsObs_CMS, HiggsObs_CMS_Long} is highly suppressed due to the Yukawa-like nature of its coupling strength to fermions, the associated production of a Higgs boson and DM particles
%,referred to as ``mono-h'' final state, 
can occur if the
Higgs boson takes part in the interaction producing the DM particles~\cite{monoHiggs3,2HDM,PhysRevD.89.075017}.
Such a production mechanism would allow to directly probe the structure of the effective DM-SM coupling.

In this paper, we  present a search for DM production in association
with an SM Higgs boson that decays into a bottom quark-antiquark pair ($\mathrm{b\bar{b}}$). As the $\Ph\rightarrow \mathrm{b\bar{b}}$ decay mode has the largest branching fraction of all Higgs boson decay modes allowed in the SM, it provides the largest signal yield. The search is performed using the data set collected by the CMS experiment~\cite{CMSdetector} at the CERN LHC at a center-of-mass energy of 13\TeV~in 2016, corresponding to an integrated luminosity of approximately 35.9\fbinv. 
Similar searches have been conducted at the LHC by both the ATLAS and the CMS Collaboration, analyzing data collected at 8~\cite{PhysRevLett.115.131801} and 13 TeV~\cite{PhysRevLett.119.181804,1807.02826}.
Results are interpreted in terms of two simplified models predicting this signature. The first one is type-2 two Higgs doublet model extended by an additional light pseudoscalar boson $a$ (2HDM+$a$)~\cite{Bauer2017}. The $a$ boson mixes with the scalar and pseudoscalar partners of the SM Higgs boson, and decays into a pair of DM particles,  $\chi\bar{\chi}$. The second model is a baryonic $\cPZpr$ model (baryonic Z')~\cite{PhysRevD.89.075017} where a vector mediator $\cPZpr$ is exchanged in the $s$-channel, radiates a Higgs boson, and subsequently decays into two DM 
particles. Representative Feynman diagrams for the two models are presented in Fig.~\ref{feyns}.


In the 2HDM+$a$ model, the DM particle candidate $\chi$ is a fermion that can couple to SM particles only through a spin-0, pseudoscalar mediator. Since the couplings of the new spin-0
mediator to SM gauge bosons are strongly suppressed, the 2HDM+$a$ model
is consistent with the measurements of the SM Higgs boson production and
decay modes, which so far show no significant deviation from SM predictions~\cite{Khachatryan:2016vau}. In contrast to previously explored 2HDM models~\cite{2HDM,Aaboud:2017yqz,Sirunyan:2017hnk}, the 2HDM+$a$ framework ensures gauge invariance and renormalizability. In this model, there are six mass eigenstates:
a light neutral charge-parity (CP)-even scalar \Ph, assumed to be the
observed 125\GeV Higgs boson, and a heavy neutral CP-even scalar \PH, that are the result of the mixing of the neutral CP-even weak eigenstates with the corresponding mixing angle $\alpha$;
a heavy neutral CP-odd pseudoscalar $A$ and a light neutral CP-odd pseudoscalar $a$, that are the result of the mixing of the CP-odd mediator $P$ with the CP-odd Higgs, with $\theta$ representing the associated mixing angle; and two heavy charged scalars \Hpm with identical mass. 

The masses of the two CP-odd Higgs bosons, the angle $\theta$ , and the ratio of vacuum
expectation values of the two CP-even Higgs bosons $\tan\beta$ are varied in this search. 
Perturbativity and unitarity put restrictions on the magnitudes and the
signs of the three quartic couplings
$\lambda_3,~\lambda_{\mathrm{P}1},~\lambda_{\mathrm{P}2}$,
and we therefore set their values to $\lambda_3=\lambda_{\mathrm{P}1}=\lambda_{\mathrm{P}2}=3$~\cite{Bauer2017}. The masses of the charged Higgs bosons and of the heavy CP-even Higgs boson are assumed to be the same as the mass of the heavy pseudoscalar, i.e., $m_{\PH}=m_{\Hpm}=m_{A}$. When an $m_{A}$ scan is performed, assumptions on $\tan\beta$ to be 1 and $\sin\theta$ to be 0.35 are made. The DM particle $\chi$ is assumed to have a mass of $m_\chi=10\GeV$.


The baryonic Z' model~\cite{PhysRevD.89.075017} is an extension of the SM with an additional U(1)$_{B}$ Z' gauge 
boson that couples to the baryon number $B$. The model predicts the existence of a new baryonic Dirac fermion that is neutral under SM gauge symmetries and stable due to the corresponding U(1)$_{B}$ symmetry. The state therefore serves as a good DM candidate.
To generate the  \cPZpr\ mass, a ``baryonic Higgs'' scalar is introduced to 
spontaneously break the U(1)$_B$ symmetry. Analogous to the SM, there remains 
a physical baryonic Higgs particle, h$_{B}$, with a coupling  $g_{\mathrm{h}_{B}\mathrm{Z'Z'}}$ to the Z' boson
and vacuum expectation value $v_{B}$. 
The \cPZpr\ and SM Higgs boson, h, interact with a coupling strength of 
$g_{\mathrm{h}_{B}\mathrm{Z'Z'}}$ = $m_{\cPZpr}^{2} \sin \theta/v_{B}$, where $\theta$ is the h-h$_{B}$ 
mixing angle. The chosen value for the \cPZpr\ coupling to quarks,
$g_\text{q}$, is 0.25 and the \cPZpr\ coupling to DM, $g_\chi$, is set to 1. This is well below the bounds $g_\text{q},g_\chi\sim4\pi$ where perturbativity and the validity of the effective field theory break down~\cite{PhysRevD.89.075017}. Constraints on the SM Higgs boson properties make the mixing angle $\theta$ consistent with $\cos\theta$ = 1 within order of 10\% uncertainties, thereby requiring  $\sin\theta$ to be less then 0.4~\cite{PhysRevD.89.075017}. In this search, $\theta$ is assumed to have $\sin\theta= 0.3$. It is also assumed that $g_{\text{h}\cPZpr\cPZpr}/m_{\mathrm{Z}'}=1$, which implies $v_B=m_{\cPZpr}\sin\theta$. This choice maximizes the cross section without violating the bounds imposed by SM measurements. The free parameters in the model under these assumptions are thus $m_{\cPZpr}$ and $m_\chi$, which are varied in this search.

\begin{figure}
\centering
 \subfloat{\includegraphics[width=0.4\textwidth]{figures/Feyn-2HDMa.pdf}}\hspace{1cm}
 \subfloat{\includegraphics[width=0.34\textwidth]{figures/Feyn-Baryonic.pdf}} \\
\caption{Feynman diagrams for the 2HDM+$a$ model (left) and the baryonic Z' model (right).}
\label{feyns}
\end{figure}


%The quantity \ptvecmiss, calculated as the negative vectorial sum of the transverse momentum (\pt) of all objects identified in an event, 
%represents the total
%momentum carried by the DM particles.
%The magnitude of this vector is referred to as \MET.
%For a given value of \mzp, the \pt of the \Az decreases as its mass increases.
%Therefore, the \MET spectrum softens with increasing \Az masses.
%A comparison of the \MET distributions expected from representative scenarios of the \cPZpr-2HDM model and the \cPZpr-Baryonic model are presented in Fig.~\ref{fig:met_signals}.


%\begin{figure}
%\centering
%\includegraphics[width=0.55\textwidth]{figures/puppimet_signals.pdf}
%\includegraphics[width=0.45\textwidth]{figures/ZpBaryonicModel.pdf}
%\caption{Reconstructed \MET for representative scenarios of two $\mathrm{h}+\mathrm{DM}$ models investigated. Coupling parameters are chosen as mentioned in the text. The \cPZpr-2HDM model in general has a significant harder \MET~spectrum than the \cPZpr-Baryonic model, which makes the former easier to distinguish from SM background processes.}
%\label{fig:met_signals}
%\end{figure}

Signal events are characterized by a large imbalance in the transverse
momentum (or hadronic recoil), which indicates the presence of invisible
DM particles, and by hadronic activity consistent with the production
of an SM Higgs boson that decays into a $\mathrm{b\bar{b}}$ pair. Thus, our search
strategy is to impose requirements on the mass of the reconstructed
Higgs boson candidate, together with the identification of the
products of hadronization of the two b quarks produced in the Higgs
boson decay, to define a data sample that is expected to be enriched
in signal events. Several different SM processes can mimic this topology, the most important of which are the top quark pair production and the production of a vector boson (V) in association with multiple jets. Statistically independent data samples are used to predict the hadronic recoil distribution for these SM processes that constitute the largest sources of background.
Both signal and background contributions to the data are extracted with a likelihood fit to the hadronic recoil distribution, performed simultaneously in all the different analysis subsamples.
