Results are interpreted in terms two of simplified models predicting
this signature. The first is a Type-2 Two Higgs Doublet Model (2HDM) extended by an additional light pseudoscalar boson a (2HDM+a)~\cite{Bauer2017}, which mixes with the scalar and pseudoscalar partners of the SM Higgs boson and decays into two DM particles, $\chi$, $\bar{\chi}$. The second is a baryonic $\cPZpr$ model (Baryonic Z')~\cite{PhysRevD.89.075017} where a vector mediator $\cPZpr$ is exchanged in the s-channel, radiates a Higgs boson, and subsequently decays into two DM 
particles. Representative Feynman diagrams for the two models are presented in Fig.~\ref{feyns}.


In the 2HDM+a model, the DM particle candidate $\chi$ is a SM singlet
fermion and thus can only couple to SM particles through a
pseudoscalar, spin-0, mediator. Since the couplings of the new spin-0
mediator to SM gauge bosons are strongly suppressed, the 2HDM+a model
is consistent with  measurements of the SM Higgs boson production and
decay modes, which so far show no significant deviation from SM predictions~\cite{Khachatryan:2016vau}. In contrast to previously explored 2HDM models~\cite{2HDM,Aaboud:2017yqz,Sirunyan:2017hnk}, the 2HDM+a framework ensures gauge invariance and renormalizability. In this model, there are six physical Higgs bosons:
a light neutral CP-even scalar \Ph, assumed to be the
observed 125\GeV Higgs boson; a heavy neutral CP-even scalar \PH;
a heavy neutral CP-odd pseudoscalar \Az; a light neutral CP-odd pseudoscalar a; and two heavy charged scalars \Hpm. 

%Mass hypotheses for the \cPZpr\ resonance and for the \Az particle are
%considered between 600 and 4000\GeV and between 300 and 800\GeV, respectively. The mass of the DM particles $m_{\chi}$
%is assumed to be less than or equal to 100\GeV. The 
%ratio of the vacuum expectation values of the two Higgs fields coupling to the up-type and down-type
%quarks, $\tan\beta$, and the coupling of the \Az particle to DM
%particles, $g_{\chi}$, are both fixed at unity. 
%Hypotheses for the mass of the \Az particle to be less than 300\GeV are excluded by constraints on flavor changing
%neutral currents from measurements of $\cPqb\rightarrow \cPqs\gamma$ \cite{Branco:2011iw},
%and therefore are not considered. 
%With the assumed dark matter particle mass, the value of $\mathcal{B}(\Az\to\chi\overline{\chi})$ is $\approx$ 100\% 
%for $\maz = 300\GeV$. 
%The branching fraction starts to decrease for $\maz$ greater than twice the mass
%of the top quark as the decay $\Az\to$ $\cPqt\cPaqt$ becomes kinematically
%accessible. 

The following parameters are varied in this search: the masses of the
two CP-odd Higgs bosons, the angle $\theta$ associated with the mixing
between the two CP-odd Higgs bosons, and the ratio of vacuum
expectation values of the two CP-even Higgs bosons $\beta$.
Perturbativity and unitarity put restrictions on the magnitude and the
signs of the three quartic couplings
$\lambda_3,~\lambda_{P1},~\lambda_{P2}$ to be of the order of unity,
and we therefore set their values as $\lambda_3=\lambda_{P1}=\lambda_{P2}=3$~\cite{Bauer2017}. Masses of the charged Higgs bosons and of the heavy CP-even Higgs boson are assumed to be the same as the mass of the heavy pseudoscalar, i.e., $m_{\PH}=m_{\Hpm}=m_{\Az}$. The DM particle $\chi$ is assumed to have a mass of $m_\chi=10\GeV$.


The Baryonic Z' model~\cite{PhysRevD.89.075017} is an extension of the SM and 
assumes that the baryon number $B$ is gauged, with the Z' being the gauge 
boson of U(1)$_{B}$. The model predicts the existence of a new baryonic state that is neutral under SM gauge symmetries and stable due to the corresponding U(1)$_{B}$ symmetry. The state therefore serves as a good DM candidate.
To generate the  \cPZpr\ mass, a ``baryonic Higgs'' scalar is introduced to 
spontaneously break the U(1)$_B$ symmetry. Analogous to the SM, there remains 
a physical baryonic Higgs particle, h$_{B}$, with a coupling of h$_{B}$Z'Z' 
and vacuum expectation value of $v_{B}$. 
The \cPZpr\ and SM Higgs boson, h, interact with a coupling strength of 
$g_{\text{h}\cPZpr\cPZpr} = m_{\cPZpr}^{2} \sin \theta/v_{B}$, where $\theta$ is the h-h$_{B}$ 
mixing angle. The chosen value for the \cPZpr\ coupling to quarks,
$g_\text{q}$, is 0.25 and the \cPZpr\ coupling to DM, $g_\chi$, is set to 1. This is well below the bounds $g_\text{q},g_\chi\sim4\pi$ where perturbativity and the validity of the effective field theory break down~\cite{PhysRevD.89.075017}. The mixing angle $\sin\theta$ is assumed to have $\sin\theta= 0.3$. It is also assumed that $g_{\text{h}\cPZpr\cPZpr}/m_{\mathrm{Z}'}=1$, which implies $v_B=m_{\cPZpr}\sin\theta$. This choice maximizes the cross section without violating the bounds. The free parameters in the model under these assumptions are thus $m_{\cPZpr}$ and $m_\chi$, which are varied in this search.

\begin{figure}
\centering
 \subfloat{\includegraphics[width=0.4\textwidth]{figures/Feyn-2HDMa.pdf}}\hspace{1cm}
 \subfloat{\includegraphics[width=0.34\textwidth]{figures/Feyn-Baryonic.pdf}} \\
\caption{Feynman diagrams for the 2HDM+a model (left) and the Baryonic Z' model (right).}
\label{feyns}
\end{figure}


%The quantity \ptvecmiss, calculated as the negative vectorial sum of the transverse momentum (\pt) of all objects identified in an event, 
%represents the total
%momentum carried by the DM particles.
%The magnitude of this vector is referred to as \MET.
%For a given value of \mzp, the \pt of the \Az decreases as its mass increases.
%Therefore, the \MET spectrum softens with increasing \Az masses.
%A comparison of the \MET distributions expected from representative scenarios of the \cPZpr-2HDM model and the \cPZpr-Baryonic model are presented in Fig.~\ref{fig:met_signals}.


%\begin{figure}
%\centering
%\includegraphics[width=0.55\textwidth]{figures/puppimet_signals.pdf}
%\includegraphics[width=0.45\textwidth]{figures/ZpBaryonicModel.pdf}
%\caption{Reconstructed \MET for representative scenarios of two $\mathrm{h}+\mathrm{DM}$ models investigated. Coupling parameters are chosen as mentioned in the text. The \cPZpr-2HDM model in general has a significant harder \MET~spectrum than the \cPZpr-Baryonic model, which makes the former easier to distinguish from SM background processes.}
%\label{fig:met_signals}
%\end{figure}

The Higgs boson recoils against DM particles and is expected to be
have a large Lorentz boost. It decays into a pair of b quarks, which
form collimated sprays of hadrons that can be
reconstructed as jets. The DM particles escape from the detector and
lead to a measurable imbalance in transverse momentum in the
reconstructed event. Given these signatures, the main background processes in
this search are top quark pair production and the production of a vector boson (V) in association with multiple jets. %Data in control regions (CRs) enriched in the respective background processes are used to derive predictions for the SM background.
